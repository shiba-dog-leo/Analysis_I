\documentclass[12pt, dvipdfmx, fleqn]{jarticle}%日本語,[dvipdfmx]は全てのパッケージでdvipdfmxを指定するために必要
%\usepackage[top=15truemm, bottom=20truemm]{geometry}%余白調整
\usepackage{ascmac}%枠囲み
\usepackage{amsmath}%文中の分数がつぶれないようにするため
\usepackage{amssymb}%白抜き文字
\usepackage{enumerate}%箇条書きのカウンタを整えるため
\usepackage{framed,color}%影付けを行うため
\usepackage{wrapfig}%図の回り込み
\usepackage{physics}%数式を軽量に
\usepackage{tikz}%グラフの描画
\usetikzlibrary{intersections, calc}%tikzの補助


\setlength{\mathindent}{60pt}%数式左揃え

\definecolor{shadecolor}{gray}{0.85}%1.00に近づくほど灰色に近づく(明るくなる)

\definecolor{lg}{rgb}{0.75,0.75,0.75} %lightgray色の定義
\newenvironment{lgbar}{%
  \def\FrameCommand{\textcolor{lg}{\vrule width 3pt} \hspace{10pt}}%
  \MakeFramed {\advance\hsize-\width \FrameRestore}}%
{\endMakeFramed}

%\newtheorem{thm}{定理}[part]%定理番号はpartごと
%\newtheorem{de}{定義}[section]
%\newtheorem{prob}{問題}[section]
%\newtheorem{ex}{例}[section]
%\newtheorem{cor}{系}[section]
%\newtheorem{prop}{命題}[section]
%\newtheorem{fact}{事実}[section]
%\newtheorem{rem}{Remark}[section]
%\newtheorem{lem}{補題}[section]
\newtheorem{ans}{解答}%[subsection]
\renewcommand{\theans}{}%番号を表示しない
%参考<https://www.lightstone.co.jp/latex/kb0078.html>

\parskip=10pt plus 1pt%段落間隔調整

%%%%%%%%%%%%%%%%% END OF PREAMBLE %%%%%%%%%%%%%%%%